\documentclass[UTF8]{ctexart}
\usepackage{geometry}
\geometry{margin=1.5cm, vmargin={0pt,1cm}}
\setlength{\topmargin}{-1cm}
\setlength{\paperheight}{29.7cm}
\setlength{\textheight}{25.3cm}

% useful packages.
\usepackage{amsfonts}
\usepackage{amsmath}
\usepackage{amssymb}
\usepackage{amsthm}
\usepackage{enumitem} % 使用 enumitem 包代替 enumerate
\usepackage{graphicx}
\usepackage{multicol}
\usepackage{fancyhdr}
\usepackage{layout}
\usepackage{listings}
\usepackage{float, caption}

\lstset{
    basicstyle=\ttfamily, basewidth=0.5em
}

% some common command
\newcommand{\dif}{\mathrm{d}}
\newcommand{\avg}[1]{\left\langle #1 \right\rangle}
\newcommand{\difFrac}[2]{\frac{\dif #1}{\dif #2}}
\newcommand{\pdfFrac}[2]{\frac{\partial #1}{\partial #2}}
\newcommand{\OFL}{\mathrm{OFL}}
\newcommand{\UFL}{\mathrm{UFL}}
\newcommand{\fl}{\mathrm{fl}}
\newcommand{\op}{\odot}
\newcommand{\Eabs}{E_{\mathrm{abs}}}
\newcommand{\Erel}{E_{\mathrm{rel}}}

\begin{document}

\pagestyle{fancy}
\fancyhead{}
\lhead{徐磊, 3230101707}
\chead{数据结构与算法第六次作业}
\rhead{Nov.11th, 2024}

\section{标准二叉搜索树删除过程}

删除操作首先遵循二叉搜索树的基本规则,根据节点的情况分为三种情况处理:
   1.叶节点删除:如果目标节点没有子节点,直接删除该节点。
   2.单子节点删除:如果节点只有一个子节点,则用该子节点替代当前节点。
   3.双子节点删除:如果节点有两个子节点,则选择右子树中的最小节点(即中序后继)替代该节点的值,并递归删除该最小节点。此操作确保树的有序性。

\section{更新节点高度}

删除操作完成后,沿路径回溯至根节点,并更新所有祖先节点的高度。节点的高度定义为 $1 + \max(\text{左子树高度}, \text{右子树高度})$,因此在子树发生变化时,必须重新计算所有相关祖先节点的高度。

\section{重新平衡 AVL 树}

在更新高度后,某些节点的平衡因子可能会失衡(即左右子树的高度差超过 1)。此时,需要对失衡节点进行旋转操作来恢复平衡。常见的旋转方式包括:
   单右旋转:当节点的左子树的左侧过高时,执行右旋转。
   单左旋转:当节点的右子树的右侧过高时,执行左旋转。
   双旋转(左-右旋转):当节点的左子树的右侧过高时,首先对左子节点执行左旋转,再对当前节点执行右旋转。
   双旋转(右-左旋转):当节点的右子树的左侧过高时,首先对右子节点执行右旋转,再对当前节点执行左旋转。

\section{结论}

通过实现上述删除和旋转步骤的 `remove` 函数,可以在 AVL 树中有效删除节点并保持树的平衡。

\end{document}

%%% Local Variables:
%%% mode: latex
%%% TeX-master: t
%%% End: