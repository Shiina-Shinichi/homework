\documentclass[Unicode]{ctexart}
\usepackage{geometry, CJKutf8}
\geometry{margin=1.5cm, vmargin={0pt,1cm}}
\setlength{\topmargin}{-1cm}
\setlength{\paperheight}{29.7cm}
\setlength{\textheight}{25.3cm}

% useful packages.
\usepackage{amsfonts}
\usepackage{amsmath}
\usepackage{amssymb}
\usepackage{amsthm}
\usepackage{enumerate}
\usepackage{graphicx}
\usepackage{multicol}
\usepackage{fancyhdr}
\usepackage{layout}
\usepackage{listings}
\usepackage{float, caption}

\lstset{
    basicstyle=\ttfamily, basewidth=0.5em
}

% some common command
\newcommand{\dif}{\mathrm{d}}
\newcommand{\avg}[1]{\left\langle #1 \right\rangle}
\newcommand{\difFrac}[2]{\frac{\dif #1}{\dif #2}}
\newcommand{\pdfFrac}[2]{\frac{\partial #1}{\partial #2}}
\newcommand{\OFL}{\mathrm{OFL}}
\newcommand{\UFL}{\mathrm{UFL}}
\newcommand{\fl}{\mathrm{fl}}
\newcommand{\op}{\odot}
\newcommand{\Eabs}{E_{\mathrm{abs}}}
\newcommand{\Erel}{E_{\mathrm{rel}}}

\begin{document}

\pagestyle{fancy}
\fancyhead{}
\lhead{徐磊, 3230101707}
\chead{数据结构与算法第五次作业}
\rhead{Nov.4th, 2024}

  
\section{简介}
本文档描述了在二叉搜索树(BST)类中优化 \texttt{remove} 函数的过程。原始实现的 \texttt{remove} 函数在处理节点删除时效率较低,因为它涉及递归删除和节点内容复制。经过优化后的 \texttt{remove} 函数通过指针修改和节点替换的方式实现删除,避免了递归删除和内容复制,从而提高了效率。

\section{优化后的删除函数实现}
为了提高效率,新增了辅助函数 \texttt{detachMin}。该函数的作用是查找右子树中的最小节点并将其从子树中移除。当删除具有两个子节点的节点时,通过直接使用 \texttt{detachMin} 函数获得右子树的最小节点并替换被删除的节点,从而避免了数据复制和递归操作。

更新后的 \texttt{remove} 函数的工作流程如下:
\begin{itemize}
    \item 如果目标节点是叶子节点,则直接删除。
    \item 如果目标节点只有一个子节点,则用其子节点替换该节点。
    \item 如果目标节点有两个子节点,则调用 \texttt{detachMin} 函数获取右子树的最小节点,并用该节点替换目标节点。
\end{itemize}

优化后的 \texttt{remove} 和 \texttt{detachMin} 函数的代码如下:

\begin{verbatim}
BinaryNode* detachMin(BinaryNode*& t) {
    if (t->left == nullptr) {
        BinaryNode* minNode = t;
        t = t->right;
        return minNode;
    }
    return detachMin(t->left);
}

void remove(const Comparable& x, BinaryNode*& t) {
    if (t == nullptr) return;

    if (x < t->element) {
        remove(x, t->left);
    } else if (x > t->element) {
        remove(x, t->right);
    } else {
        if (t->left != nullptr && t->right != nullptr) {
            BinaryNode* minNode = detachMin(t->right);
            minNode->left = t->left;
            minNode->right = t->right;
            delete t;
            t = minNode;
        } else {
            BinaryNode* oldNode = t;
            t = (t->left != nullptr) ? t->left : t->right;
            delete oldNode;
        }
    }
}
\end{verbatim}

\section{测试用例及结果}

通过多种测试用例验证了优化后的 \texttt{remove} 函数的正确性和效率。以下是测试用例及其结果:

\subsection{测试用例1:删除叶子节点}
删除叶子节点 20,验证删除操作是否能正确进行,且不会影响其他节点。

\subsection{测试用例2:删除只有一个子节点的节点}
删除只有一个子节点的节点 30(其子节点为 40),验证该节点能被其唯一子节点正确替换。

\subsection{测试用例3:删除具有两个子节点的节点}
删除具有两个子节点的节点 50(左右子节点分别为 30 和 70)。使用右子树的最小节点(60)替换被删除的节点,验证二叉搜索树的结构是否完整。

\subsection{测试用例4:删除不存在的节点}
尝试删除不存在的节点 90,验证在这种情况下树结构保持不变。

\subsection{测试用例5:逐步删除所有节点}
依次删除所有节点,最终使树为空,验证删除操作的鲁棒性和效率。

测试结果表明,优化后的 \texttt{remove} 函数在各种情况下均能正确执行,并且在性能上有明显提升。

\section{结论}
本文通过优化二叉搜索树的 \texttt{remove} 函数,减少了递归调用和节点内容复制操作。新的实现显著提高了节点删除操作的效率,并通过综合测试验证了其正确性和鲁棒性。这种改进对于需要频繁删除操作的应用场景具有重要意义。

\end{document}

%%% Local Variables: 
%%% mode: latex
%%% TeX-master: t
%%% End: 
