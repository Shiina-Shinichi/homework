\documentclass{article}
\usepackage{ctex} % 支持中文
\usepackage{booktabs} % 表格美化
\usepackage{amsmath} % 数学公式
\usepackage{graphicx} % 图片插入
\usepackage{fancyhdr}

\begin{document}

\pagestyle{fancy}
\fancyhead{}
\lhead{徐磊, 3230101707}
\chead{数据结构与算法 堆排序算法作业}
\rhead{Dec.2nd, 2024}

\section{引言}
使用 \texttt{HeapSort} 函数进行排序,并使用 C++ \texttt{chrono} 库记录所花费的时间。
3. 使用 \texttt{check} 函数验证排序是否正确。
4. 使用 \texttt{std::sort\_heap()} 对相同序列进行排序,并记录其执行时间,进行性能对比。

堆排序的时间复杂度是 $O(\log n)$,因此整体的时间复杂度为 $O(n \log n)$。

\section{测试结果}
下表展示了四种序列类型的排序时间。每个序列包含 1000000 个元素,时间单位为秒。

\begin{table}[h!]
\centering
\caption{不同序列类型的排序时间}
\begin{tabular}{ccc}
\toprule
序列类型 & 堆排序时间 (秒) & \texttt{std::sort\_heap} 时间 (秒) \\
\midrule
随机序列 & 0.075565 & 0.0181685 \\
有序序列 & 0.0387278 & 0.0240654 \\
逆序序列 & 0.0413297 & 0.0244835 \\
重复元素序列 & 0.00164248 & 0.011722 \\
空序列 & 6.1e-08 & 3.8e-08 \\
单元素序列 & 2.3e-08 & 2.2e-08 \\
\bottomrule
\end{tabular}
\end{table}

\section{时间复杂度与性能分析}
\subsection{时间复杂度}
堆排序的时间复杂度由堆的构建和堆调整阶段主导。堆的构建需要 $O(n)$ 时间,而每次堆调整的时间是 $O(\log n)$。因此,堆排序的总体时间复杂度为 $O(n \log n)$。

\subsection{性能比较}
在比较堆排序与 \texttt{std::sort\_heap()} 的性能时,我们观察到以下几点:
- 在随机序列上,堆排序和 \texttt{std::sort\_heap()} 的性能差异较大,堆排序的时间明显长于 \texttt{std::sort\_heap()}。
- 在有序序列和逆序序列上,堆排序的表现良好,时间与 \texttt{std::sort\_heap()} 非常接近。
- 在重复元素序列上,性能差异较大,堆排序的时间明显低于 \texttt{std::sort\_heap()}。
- 在空序列和单元素序列上,两者的时间差异非常小,接近零。

性能差异的原因可能与 \texttt{std::sort\_heap()} 的优化实现有关。标准库中的 \texttt{std::sort\_heap()} 在实现上进行了优化,可能在某些情况下表现得更高效。然而,对于大部分实际应用,堆排序的性能与 \texttt{std::sort\_heap()} 相当。

\subsection{边界情况分析}
对于非常小的序列(例如空序列和单元素序列),两种排序算法的执行时间都非常小,几乎可以忽略不计。这些情况对于堆排序和 \texttt{std::sort\_heap()} 来说都没有实质性的影响。

\section{结论}
堆排序是一种高效的排序算法,时间复杂度为 $O(n \log n)$。它在不同类型的序列上表现良好,能够生成正确的排序结果。在与 \texttt{std::sort\_heap()} 进行性能比较时,堆排序的性能略微逊色,但差异并不显著。在实际应用中,堆排序是一个可靠且高效的排序算法,适用于大规模数据排序,尤其是在需要排序可变数据的情况下。

通过本次测试和性能评估,确认堆排序在不同场景下的稳定性和效率,未来可以通过进一步优化堆的构建阶段来提高性能。

\end{document}
