\documentclass{article}
\usepackage{ctex} % 支持中文
\usepackage{booktabs} % 表格美化
\usepackage{amsmath} % 数学公式
\usepackage{graphicx} % 图片插入
\usepackage{fancyhdr}

\title{四则运算表达式求值程序设计与测试}
\author{徐磊 3230101707}
\date{2024年12月20日}

\begin{document}

\maketitle

\section{引言}

本报告介绍了一个四则运算表达式求值程序的设计与实现,该程序能够处理包括加法、减法、乘法、除法、括号、负数以及科学计数法的中缀表达式。程序的设计核心是将中缀表达式转换为后缀表达式并进行计算,同时对输入的表达式进行合法性检查,确保括号匹配、负号和减法符号区分清晰。

\section{设计思路}

本项目使用了经典的栈(stack)数据结构来处理中缀表达式的转换和运算。在实现过程中,主要分为以下几个步骤:

\begin{enumerate}
    \item \textbf{合法性检查:}检查输入表达式是否符合基本的数学运算规则,确保括号匹配、运算符位置合法等。
    \item \textbf{中缀转后缀:}通过栈将中缀表达式转换为后缀表达式,方便后续计算。转换过程中需要正确处理运算符优先级和括号。
    \item \textbf{后缀表达式计算:}通过栈进行后缀表达式的计算,计算过程中遇到运算符时弹出栈顶的两个操作数,计算结果后推入栈中,最终栈中剩余的单一元素即为表达式的计算结果。
    \item \textbf{负号处理:}支持一元负号的正确解析,区分负号和减法运算符,特别是在连续负号(如 `1--2`)的情况下,程序能正确识别为加法。
    \item \textbf{科学计数法支持:}支持科学计数法格式的数字输入,例如 `1e2` 表示 `100`。
    \item \textbf{浮点数计算:}程序能够正确计算包含浮点数的表达式,支持高精度的运算。
    \item \textbf{括号匹配:}支持多重括号匹配,确保括号的配对和运算顺序正确。
\end{enumerate}

\section{算法实现}

\subsection{合法性检查}

在合法性检查部分,首先移除输入表达式中的所有空格,然后检查表达式中括号是否匹配、运算符是否连续、表达式是否以运算符开头或结尾、以及是否存在除零的情况。对表达式中的负号也进行了特殊处理,确保 `-` 可以作为一元负号使用。

\subsection{中缀转后缀}

使用栈来管理运算符和括号。根据运算符的优先级,决定将运算符推入栈中还是将栈中的运算符弹出到输出中。处理一元负号时,将其转化为特殊符号 `u` 以便在计算时处理。

\subsection{后缀表达式计算}

后缀表达式的计算过程是通过栈进行的,遇到操作数时将其压入栈中,遇到运算符时弹出栈顶的两个操作数,进行相应的运算后再将结果压入栈中。最后,栈中剩下的元素就是结果。

\subsection{负号处理}

负号的处理是通过判断运算符的上下文来区分一元负号和二元减法。`1--2` 被解析为 `1 + 2`,而 `1-2` 则被解析为 `1 - 2`。

\subsection{科学计数法支持}

我们通过正则表达式的方式支持科学计数法的解析,例如 `1e2` 表示 `100`,`2.5e-3` 表示 `0.0025`。

\subsection{浮点数计算}

程序支持浮点数的计算,能够精确处理带小数的数字以及运算。我们在实现中使用了 `double` 类型来确保计算精度。

\subsection{括号匹配}

为了确保表达式中多重括号的正确匹配,我们在转换过程中使用栈来检查括号的配对。在遍历表达式时遇到左括号 `(` 时将其压入栈中,遇到右括号 `)` 时从栈中弹出,确保每个右括号都有对应的左括号,并且没有多余或缺失的括号。

\section{结果分析}

在对程序进行测试时,我们设计了多种表达式来验证程序的正确性,包括合法表达式和非法表达式。以下是一些典型测试用例的结果:

\subsection{合法表达式}

\begin{itemize}
    \item \textbf{测试用例 1:} 输入 `3 + 5 * (2 - 8)`,输出 `-27`
    \item \textbf{测试用例 2:} 输入 `1 - 2`,输出 `-1`
    \item \textbf{测试用例 3:} 输入 `1--2`,输出 `3`
    \item \textbf{测试用例 4:} 输入 `(3 + 1) * 4 - 1`,输出 `15`
    \item \textbf{测试用例 5:} 输入 `-(3 + 1) * 4 - 1`,输出 `-17`
    \item \textbf{测试用例 6:} 输入 `1e2 + 3`,输出 `103`
    \item \textbf{测试用例 7:} 输入 `1e-2 + 3`,输出 `3.01`
    \item \textbf{测试用例 8:} 输入 `-2e-2 + 1`,输出 `0.98`
    \item \textbf{测试用例 9:} 输入 `3.5 + 1.5`,输出 `5`
    \item \textbf{测试用例 10:} 输入 `((3 + 5) * 2) - (1 + 1)`,输出 `14`
    \item \textbf{测试用例 11:} 输入 `(1 + (2 + 3)) * 4`,输出 `24`
    \item \textbf{测试用例 12:} 输入 `(1 + (2 * 3)) * (4 + 5)`,输出 `63`
\end{itemize}

\subsection{非法表达式}

程序还能够正确识别非法的表达式,以下是一些非法表达式的测试用例:

\begin{itemize}
    \item \textbf{测试用例 1:} 输入 `5 / 0`,输出 `ILLEGAL`(除以零错误)
    \item \textbf{测试用例 2:} 输入 `3 + a`,输出 `ILLEGAL`(包含非法字符)
    \item \textbf{测试用例 3:} 输入 `1 + ++3`,输出 `ILLEGAL`(运算符连续使用)
    \item \textbf{测试用例 4:} 输入 `1 +`,输出 `ILLEGAL`(表达式以运算符结尾)
    \item \textbf{测试用例 5:} 输入 `(3 + 5`,输出 `ILLEGAL`(括号不匹配)
    \item \textbf{测试用例 6:} 输入 `((3 + 5)`,输出 `ILLEGAL`(多重括号不匹配)
    \item \textbf{测试用例 7:} 输入 `)3 + 5(`,输出 `ILLEGAL`(括号顺序错误)
    \item \textbf{测试用例 8:} 输入 `+1 + 2`,输出 `ILLEGAL`(表达式以运算符开头)
    \item \textbf{测试用例 9:} 输入 `1++2`,输出 `ILLEGAL`(连续运算符)
\end{itemize}

\subsection{边界情况}

除了常规的测试用例,我们还测试了一些边界情况,如空表达式、无效字符的输入等:

\begin{itemize}
    \item \textbf{测试用例 1:} 输入空字符串 `""`,输出 `ILLEGAL`
    \item \textbf{测试用例 2:} 输入非法字符 `1 + @`,输出 `ILLEGAL`
    \item \textbf{测试用例 3:} 输入连续的运算符 `1++2`,输出 `ILLEGAL`
    \item \textbf{测试用例 4:} 输入含有科学计数法的负数 `-2e-2 + 1`,输出 `0.98`
\end{itemize}

\section{总结}

本报告介绍了一个四则运算表达式求值程序的设计与实现。程序通过中缀转后缀的方法解析并计算表达式,同时支持一元负号、科学计数法以及括号运算。我们通过一系列的合法性检查和测试用例验证了程序的正确性,确保它能够处理各种常见和边界情况的表达式。此外,程序能够有效识别非法表达式并给出适当的错误提示。程序也支持浮点数的高精度计算和科学计数法格式的解析,能够满足大部分基本的运算需求。

\end{document}
